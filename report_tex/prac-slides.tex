% In principle, this file can be redistributed and/or modified under
% the terms of the GNU Public License, version 2.


\documentclass{beamer}

%\usetheme{AnnArbor}
%\usetheme{Antibes}
%\usetheme{Bergen}
%\usetheme{Berkeley}
%\usetheme{Berlin}
%\usetheme{Boadilla}
%\usetheme{boxes}
%\usetheme{CambridgeUS}
%\usetheme{Copenhagen}
%\usetheme{Darmstadt}
%\usetheme{default}
%\usetheme{Frankfurt}
%\usetheme{Goettingen}
%\usetheme{Hannover}
%\usetheme{Ilmenau}
%\usetheme{JuanLesPins}
%\usetheme{Luebeck}
\usetheme{Madrid}
%\usetheme{Malmoe}
%\usetheme{Marburg}
%\usetheme{Montpellier}
%\usetheme{PaloAlto}
%\usetheme{Pittsburgh}
%\usetheme{Rochester}
%\usetheme{Singapore}
%\usetheme{Szeged}
%\usetheme{Warsaw}
\usepackage[utf8]{inputenc}

\title{Prostate Adenocaricome Insights}

\subtitle{Gene Expression approach from TCGA RNA-seq data}

\author{A. Auladell\inst{1}, J. Martí \inst{1} \& D. Mas\inst{1}}

\institute[UPF] 
{
  \inst{1}%
  Department of Experimental \& Health Science\\
  Universitat Pompeu Fabra
}

\date{\today}

\subject{IEO: Information Extraction from Omics technologies}
% This is only inserted into the PDF information catalog. Can be left
% out. 

% If you have a file called "university-logo-filename.xxx", where xxx
% is a graphic format that can be processed by latex or pdflatex,
% resp., then you can add a logo as follows:

\pgfdeclareimage[height=1cm]{upf-logo}{upf.png}
\logo{\pgfuseimage{upf-logo}}

% Delete this, if you do not want the table of contents to pop up at
% the beginning of each subsection:
\AtBeginSection[]
{
  \begin{frame}<beamer>{Outline}
    \tableofcontents[currentsection]
  \end{frame}
}

% Let's get started
\begin{document}

\begin{frame}
  \titlepage
\end{frame}

\begin{frame}{Outline}
  \tableofcontents
  % You might wish to add the option [pausesections]
\end{frame}

% Section and subsections will appear in the presentation overview
% and table of contents.
\section{Introduction}

\subsection{Prostate cancer generalities}
\begin{frame}{Introduction}{Prostate cancer generalities}
  	\begin{block}{Incidence}
		\begin{itemize}
		\item 1 in 7 men will be diagnosed during his lifespan.
		\item 2nd deadliest cancer type in males.
		\item Higher prevalence in African Ancestry.
		\end{itemize}
	\end{block}
	\pause % The slide will pause after showing the first item
	\begin{block}{Previous studies}
		\begin{itemize}
		\item Other RNA-seq studies have been published, most of them at gene level.
		\item The TCGA Consortium gathered a big cohort of patients and have performed several tests and sequencing techniques.
		\end{itemize}
	\end{block}
\end{frame}

\subsection{The Genome Cancer Atlas article}

% You can reveal the parts of a slide one at a time
% with the \pause command:
\begin{frame}{Introduction}{TGCA \cite{Abeshouse2015}}
  \begin{itemize}
  \item<1-> {
    AIM: Create a molecular taxonomy of Prostate Cancer based on 333 tumor samples
    \pause % The slide will pause after showing the first item
  }
  \item<2-> {   
	Data: Whole-exome sequencing (somatic mutations); Array-based methods (somatic copy-number changes); DNA 		methylation and mRNA sequencing (Gene Expression)  
  }
  % You can also specify when the content should appear
  % by using <n->:
  \item<3-> {
    74\% of all tumors were identified in 7 different molecular subtypes: (1) ERG, (2) ETV1, (3) ETV4, or (4) FLI1, (5) SPOP, (6) FOXA1 or (7) IDH1.
  }
  \item<4-> {
  	No relationship between clinical data and molecular cancer subtype (p.e. Gleason score).
  }
  \item<5->{
  	However, subtypes did not share any particular expression pattern. 
  }
  \end{itemize}
\end{frame}

\section{Methods}

\subsection{Data Availability \& Experimental Design}

\begin{frame}{Blocks}
\begin{block}{Block Title}
You can also highlight sections of your presentation in a block, with it's own title
\end{block}
\begin{theorem}
There are separate environments for theorems, examples, definitions and proofs.
\end{theorem}
\begin{example}
Here is an example of an example block.
\end{example}
\end{frame}

\subsection{Processing}


\subsection{Statistical Analysis}



\section{Results}


\subsection{DE genes}


\subsection{Functional Enrichment}

\subsection{Gene Set Expression Analysis}



\section{Conclusions}

\begin{frame}{Summary}
  \begin{itemize}
  \item
    The \alert{first main message} of your talk in one or two lines.
  \item
    The \alert{second main message} of your talk in one or two lines.
  \item
    Perhaps a \alert{third message}, but not more than that.
  \end{itemize}
  
  \begin{itemize}
  \item
    Outlook
    \begin{itemize}
    \item
      Something you haven't solved. 
    \item
      Something else you haven't solved.
    \end{itemize}
  \end{itemize}
\end{frame}



% All of the following is optional and typically not needed. 
\appendix
\section<presentation>*{\appendixname}
\subsection<presentation>*{For Further Reading}


\begin{frame}[allowframebreaks]%in case more than 1 slide needed
\frametitle<presentation>{References}
    {\footnotesize
    \bibliographystyle{unsrt}
    \bibliography{prac-bibliography}
    }
\end{frame}


\end{document}


