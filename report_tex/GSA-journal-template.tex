\documentclass[9pt,twocolumn,twoside]{gsajnl}
% Use the documentclass option 'lineno' to view line numbers

\articletype{inv} % article type
% {inv} Investigation
% {gs} Genomic Selection
% {goi} Genetics of Immunity
% {gos} Genetics of Sex
% {mp} Multiparental Populations


\title{Prostate Adenocarcinoma insights from a Gene Expression approach from RNA-seq data}

\author[$\ast$,$\dagger$]{Adria Auladell-Martin}
\author[$\ast$,$\dagger$]{Joan Marti-Carreras}
\author[$\ast$,$\dagger$]{David Mas-Ponte}
\author[$\ast$,1]{Robert Castelo-Valdueza}

\affil[$\ast$]{M.Sc. in Bioinformatics at Department of Experimental and Health Sciences (CEXS), Universitat Pompeu Fabra}
\affil[$\dagger$]{Authors Contributed Equally to this work}

\keywords{prostate cancer; adenocarcinoma; RNA-Seq; paired data.}

\runningtitle{GENETICS Journal Template on Overleaf} % For use in the footer

\correspondingauthor{Robert Castelo Valdueza}

\begin{abstract}
The abstract should be written for people who may not read the entire paper, so it must stand on its own. The impression it makes usually determines whether the reader will go on to read the article, so the abstract must be engaging, clear, and concise. In addition, the abstract may be the only part of the article that is indexed in databases, so it must accurately reflect the content of the article. A well-written abstract is the  most effective way to reach intended readers, leading to more robust search, retrieval, and usage of the article.

\end{abstract}

\setboolean{displaycopyright}{true}

\begin{document}

\maketitle
\thispagestyle{firststyle}
\marginmark
\firstpagefootnote
\correspondingauthoraffiliation{adria.auladell@estudiant.upf.edu, joan.marti02@estudiant.upf.edu, david.mas@estudiant.upf.edu. Corresponding author: Robert Castelo Valdueza, robert.castelo@upf.edu.}
\vspace{-11pt}%

\lettrine[lines=2]{\color{color2}T}{}his \textit{Genetics} journal template is provided to help you write your work in the correct journal format. Instructions for use are provided below.

\section*{Your Abstract}

In addition to the guidelines provided in the example abstract above, your abstract should:

\begin{itemize}
\item provide a synopsis of the entire article;
\item begin with the broad context of the study, followed by specific background for the study;
\item describe the purpose, methods and procedures, core findings and results, and conclusions of the study;
\item emphasize new or important aspects of the research;
\item engage the broad readership of GENETICS and be understandable to a diverse audience (avoid using jargon);
\item be a single paragraph of less than 250 words;
\item contain the full name of the organism studied;
\item NOT contain citations or abbreviations.
\end{itemize}


\section*{Introduction}
Prostate adenocarcinoma (PRAC), also referred as prostate cancer, is  the second most deadly cancer disease (following lung cancer), and one with the highest prevalences. As the American Cancer Society explains, about 1 man in 7 will be diagnosed with PRAC during his lifespan, and of those diagnosed about 60\% are men over 65 years. Despite it might affect younger men (under 40 years old), it is not frequent. The mean of diagnosis is around 66 years old \citep{prostatestatistics}.

For patients whose cancer has spread, their survival time is usually one to three years. It was estimated that for 2011, 240,890 men would be diagnosed and 33,720 would die from prostate cancer.


Since the first publication of the human genome, multiple international projects have been carried in order to study the genetic bases of the this pathology. One of the most prominent is The Cancer Genome Atlas Project (TCGA) \citep{tgca}. Recently, part of its information was released with part of the original data \cite{Rahman15112015}. TCGA established the following points:

\begin{itemize}
\item 74\% of all tumors being assignable to one of seven molecular classes based on distinct oncogenes drivers:
        fusions involving (1) ERG, (2) ETV1, (3) ETV4, or (4) FLI1 (46\%, 8\%, 4\%, and 1\%, respectively)
        mutations in (5) SPOP or (6) FOXA1; or (7) IDH1. (11\%, 3\%, and 1\%, respectively).
\item 25\% of the prostate cancers had a presumed actionable lesion in the PI3K or MAPK signalling pathways, and DNA repair genes were inactivated in 19\%.
\end{itemize}


The aim of this study is to provide a second outlook to the released data of the TCGA. For doing so we are to perform a differential expression analysis using paired data from the TCGA.

\section*{Materials and Methods}
The examination of the RNA counts can be performed in an average laptop. As a limiting factor for analysing the data, it is recommended to have at least 4 Gb of RAM and CPU Intel Core i5 or superior. HDD memory is not a limiting factor for this analysis.

The software used was R version 3.3.0 "Supposedly Educational" \cite{R}. Main packages where download and installed through Bioconductor version 3.3 (BiocInstaller 1.22.2)\cite{bioconductor}.

Detailed information about package versions can be found on the section "Session Info" of the Supplementary Materials.

In the following sections we describe briefly our analyses. However, a reproducible version of our analysis with all the code elements is also available in the supplementary material.

\subsection*{Data Availability}

The data sets are tables of RNA-seq counts generated by Rahman et al. \cite{Rahman15112015} from the TCGA raw sequence read data using the Rsubread/featureCounts pipeline for all data sets \cite{Rsubread}. They also processed clinical data that forms part of these data sets. PRAC dataset consist initially of 502 tumour samples and 52 normal samples. The phenotypic variable that indicates the tumour or normal status of the sample is called type.

The data was used for this analysis is a variation of the above mentioned datasets, curated by PhD. Robert Castelo-Valdueza \url{http://functionalgenomics.upf.edu/courses/IEO/projects/datasets/sePRAD.rds}.

\subsection*{Experimental Design}
From the initial dataset, there are 52 normal samples and 502 tumour samples. Searching for coincidences in the patients barcode, it was found out that the dataset had mixed paired and unpaired data. It was decided that the paired design was a better choice, as it takes better into account the differences between subjects.

After applying the corresponding logical mask, the dataset was composed by 50 normal samples and 50 tumour samples.

\subsection*{Quality Assessment and Normalization}
DGE functions were used for normalization, which belongs to edgeR package \cite{Robinson01012010}. Normalization factors were calculated and counts per million reads (logCPM) computed.

LogCPM distributions area compared between normal samples and tumour samples. There are no significant differences. Transcripts with logCPM < 1 were deprecated.

Despite there was a peak of high library size in the library size distribution, it was decided to not to take it out. The rest of the distribution was considered to be uniform.

When plotting the log ratio - mean average plots (MA-plots) for all samples, it was noticed that the normalization procedure succeeded. Any deviation was considered as non-significance.

\subsection*{Batch Effect Identification}
Batch effect was evaluated by means of the TCGA barcode information. The code specified center of analysis, tissue source site and plate of the NGS machine. The center was the University of North Carolina for all samples and the tissue source site/plate variables were equilibrated in the tumour/normal comparison (see SM). To evaluate possible anomalous samples, hierarchical clustering (HC) from \cite{pheatmap} package and multidimensional scaling (MDS) from \cite{limma} package were performed. HC, clustered by Spearman correlation, presented mainly a good clustering, with some of the paired samples clustering together due to the similarities within individual (see SM). MDS plot presented an anomalous clustering of normal samples from the same batch. These samples were filtered out, remaining the 42 tumour/normal samples.

\subsection*{Differential Expression Analysis}
Talk about p-value
\subsection*{Functional Enrichment}


The resulting DE genes from the previous analyses have been tested for functional enrichment using Gene Ontology gene sets. A hyper-geometrical test from the GOstats package \citep{GOstats} was used. As parameters, a p-value cutoff of 0.05 have been set in order to filter only the most significant results. Then, the results have been pruned again for sets with at least 8 genes to avoid extreme results with infinite OddsRatio.

The same analysis have been also performed in the DE over-expressed and under-expressed gene lists to clarify if the GO biological process is indeed enhanced or decreased in tumor cells. The aim of this stratified analysis is to improve the biological content of our results.


\subsection*{Gene Set expression analysis}
In order to analyse differential expression in gene sets and not uniquely at the gene level 2 different approaches have been used, a Gene Set Enrichment Analysis (GSEA) and a Gene Set Variation Analysis (GSVA).

For both GSEA and GSVA the gene sets from the GSVAdata package \citep{GSVAdata} have been selected. After that, the original gene sets were reduced only to curated gene sets  (C2 type) . In this case the KEGG  \citep{kanehisa2016kegg}, Reactome \citep{fabregat2016reactome} and   Biocarta \citep{nishimura2001biocarta} databases were selected. Gene Sets related with Prostate Cancer have been also included from the same data package.


The package GSEABase \citep{GSEABase} was used to define Gene Sets and match gene identifiers to our Entrez Id codes. The classic GSEA consist in the generation of an incidence matrix that let us build a Enrichment Score by that is dependent on how many genes from the gene set are ranked as most DE genes.

The GSVA package \citep{GSVA} was used to define the set expression values. Then, limma  \citep{limma} and SVA packages were used to perform the linear regression and to improve the model respectively \citep{leek2007capturing,svamanual} . Finally, the 30 top DE pathways have been clustered by sample in order to obtain a clear visualization of the ones over and under-expressed in tumor and normal samples.


\section*{Results}
\subsubsection*{Differential Expression Analysis}
JOAN, gradient de significacincia. agfem aquests pq ens en podem fiar més que els altres pq tenen més escore,pero pas x rs mes
After computing the multiple linear model test, and the eBayes test \cite{limma} 	it was found out that:
\begin{itemize}
\item 2168 genes are under-expressed in tumour samples compared to normal samples.
\item 7569 genes have no significant difference in their expression between sample types.
\item 2127 genes are over-expressed in tumour samples compared to normal samples.
\end{itemize}
\subsubsection*{Functional Enrichment}
ADRIA
\subsubsection*{Gene set expression analysis}
DAVID


\paragraph{Under-expression of Endogenous Sterols } The genes in this gene set, 


\section*{Discussion}

Creure'ns o no els resultats


\section*{Examples of Article Components}
\label{sec:examples}

The sections below show examples of different header levels, which you can use in the primary sections of the manuscript (Results, Discussion, etc.) to organize your content.

\section*{First level section header}

Use this level to group two or more closely related headings in a long article.

\subsection*{Second level section header}

Second level section text.

\subsubsection*{Third level section header:}

Third level section text. These headings may be numbered, but only when the numbers must be cited in the text.

\section*{Figures and Tables}

Figures and Tables should be labelled and referenced in the standard way using the \verb|\label{}| and \verb|\ref{}| commands.

\subsection*{Sample Figure}

Figure \ref{fig:spectrum} shows an example figure.

\begin{figure}[htbp]
\centering
\includegraphics[width=\linewidth]{Custering}
\caption{Initial Cluster of Final data set clustered by gene expression levels. Although some samples are not clustered ideally with their correspondent type, 2 groups can be distinguished clearly. The top-left group contains most of the tumor samples and the bottom-right contains most of the Normal samples. 
}%
\label{fig:Clustering}
\end{figure}

\begin{figure}[htbp]
\centering
\includegraphics[width=\linewidth]{CusteringGSVA}
\caption{Heatmap from the Clustering of the DE gene sets obtained from GSVA and samples.
}%
\label{fig:ClusteringGSVA}
\end{figure}




\subsection*{Sample Table}

Table \ref{tab:shape-functions} shows an example table. Avoid shading, color type, line drawings, graphics, or other illustrations within tables. Use tables for data only; present drawings, graphics, and illustrations as separate figures. Histograms should not be used to present data that can be captured easily in text or small tables, as they take up much more space.

Tables numbers are given in Arabic numerals. Tables should not be numbered 1A, 1B, etc., but if necessary, interior parts of the table can be labeled A, B, etc. for easy reference in the text.

% latex table generated in R 3.3.0 by xtable 1.8-2 package
% Tue Jun 14 13:44:38 2016
\begin{table*}[htbp]
\centering
\caption{\bf GO enriched terms}
\begin{tableminipage}{\textwidth}
 \begin{tabular}{l|c|r|r|r|r|r|p{3cm}|p{3cm}|}
   \hline
  & GOBPID & Pvalue & OddsRatio & ExpCount & Count & Size & Term & Genes \\
   \hline
 1 & GO:1902547 & 0.00 & Inf & 2.19 &   6 &   6 & regulation of cellular response to vascular endothelial growth factor stimulus & DAB2IP, DCN, ADGRA2, HRG, ADAMTS3, CD63 \\
   2 & GO:0045078 & 0.00 & 12.21 & 2.92 &   7 &   8 & positive regulation of interferon-gamma biosynthetic process & EBI3, CEBPG, ZFPM1, IL27, IL12B, IL21, CD276 \\
   3 & GO:2001044 & 0.00 & 12.21 & 2.92 &   7 &   8 & regulation of integrin-mediated signaling pathway & CTNNA1, EMP2, LIMS2, PRKD1, PHACTR4, ITGB1BP1, CD63 \\
   4 & GO:0002115 & 0.01 & 10.47 & 2.55 &   6 &   7 & store-operated calcium entry & SPINK1, ORAI2, CRACR2A, ORAI1, CD84, HOMER1 \\
   5 & GO:0016093 & 0.01 & 10.47 & 2.55 &   6 &   7 & polyprenol metabolic process & NUS1, DPAGT1, ALG5, AKR1B10, AKR1C3, DPM2 \\
    \hline
 \end{tabular}
 \label{tab:GOenrichment}
\end{tableminipage}
\end{table*}


\begin{table*}[htbp]
\centering
\caption{\bf Students and their grades}
\begin{tableminipage}{\textwidth}
\begin{tabularx}{\textwidth}{XXXX}
\hline
Student & Grade\footnote{This is an example of a footnote in a table. Lowercase, superscript italic letters (a, b, c, etc.) are used by default. You can also use *, **, and *** to indicate conventional levels of statistical significance, explained below the table.} & Rank & Notes \\
\hline
Alice & 82\% & 1 & Performed very well.\\
Bob & 65\% & 3 & Not up to his usual standard.\\
Charlie & 73\% & 2 & A good attempt.\\
\hline
\end{tabularx}
  \label{tab:shape-functions}
\end{tableminipage}
\end{table*}

\bibliography{prac-bibliography}

\end{document}
